\documentclass{article}
\usepackage[latin1]{inputenc}

\author{Federico Bugni, Federico Paulovsky, Gervasio Perez}
\title{Análisis Automático de Programas}
\title{Trabajo práctico Nº 1: Verificación de Programas}

\begin{document}

\maketitle

\section{Ejercicio 1}

Este ejercicio nos sirvió mucho para empezar a utilizar CVC3. La resolución se encuentra en el archivo \textit{Ejercicio1} . Utilizamos fuertemente los \textit{FOR ALL} y la definición de variables; que esto ultimo lo vamos a utilizar seguido mas adelante.  

\section{Ejercicios 2 y 3}

En esta sección trataremos los aspectos del TP relacionados a la implementación del verificador Pest.

\subsection{Consideraciones generales de diseño}

En nuestro TP utilizamos extensivamente el patrón \textit{Visitor} para las recorridas de los árboles sintácticos de programas Pest, de predicados lógicos y de términos.

\subsection{Desarrollo}

\subsubsection{Primer enfoque de verificación}

Inicialmente optamos por implementar el verificador armando una fórmula lógica con cuantificadores, lo cual probó ser arduo especialmente con respecto al \textit{debugging}.
Luego de la clase práctica del Miércoles 14/9 decidimos cambiar el enfoque completamente, convirtiendo el output CVC3 en una secuencia de declaraciones de variables, y comandos ASSERT y comandos QUERY para verificar condiciones. El cambio no resultó problemático porque recién estábamos llegando a implementar el if y buena parte del código de los Visitors pudo ser reutilizada.

\subsubsection{Versión final}

Para el nuevo enfoque adoptamos el concepto de \textit{contexto} mencionado en clase. 

\subsection{Clases desarrolladas}

Para la realización del TP desarrollamos los paquetes \textbf{budapest.pest.pesttocvc3} y \textbf{budapest.pest.predtocvc3} que contienen el código de traducción. Éstos contienen las siguientes clases:
\begin{itemize}
\item PestToCVC3Translator: Implementación de Visitor que recorre un programa Pest y lo traduce, devolviendo un String con los comandos CVC3 que lo verifican.
\item PestVarContext: Representación de un contexto de variables Pest versionadas. 
\item PredVarReplacer y TrmVarReplacer: Visitors que realizan el reemplazo de variables en Predicados y en Terms respectivamente. Hacen uso de un PestVarContext de donde toman el nombre que debe usarse para cada variable en el reemplazo.
\item PredParamReplacer y TrmParamReplacer: Similares a los anteriores, pero hacen un reemplazo de un nombre de variable por un Term cualquiera. Necesarios para implementar el reemplazo de variables en una llamada a función.
\item PestVarBinder: Auxiliar para calcular el mapeo de nombres de parámetro de un procedimiento a las expresiones que se usaron para invocarlo.
\end{itemize}

\subsubsection{PestToCVC3Translator}

Este Visitor hace uso de contextos de variables para resolver el nombre que debe tener cada "instancia" de una variable al momento de usarla en una fórmula CVC.

\end{document}